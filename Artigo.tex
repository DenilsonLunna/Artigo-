\documentclass[12pt]{article}
\usepackage[top=3cm,bottom=2cm,left=3cm,right=2cm]{geometry}
\usepackage[utf8]{inputenc}
\usepackage[english,brazil]{babel}




\begin{document}
	
	\title{\textbf{Uso de jogos digitais no ensino}}
	\author{\textit{Denilson Sousa}\\\textit{Ismael Moreira}}
	
	\maketitle % Data

	\selectlanguage{brazil}
	\begin{abstract}
		Este artigo tem como objetivo mostrar a eficiência do uso de jogos digitais no ensino. Para ele desenvolvemos um jogo chamado \textit{Run Aedes} que tem como objetivo informar sobre o \textit{Aedes Aegypti}. Aplicamos o jogo em uma escola da nossa cidade e colhemos alguns dados importantes e curiosos sobre o comportamento das pessoas (jovens) no uso do software.
	
		\textbf{Palavras-chave}: Jogos Educativos; Aedes Aegypti; Jogos Digitais;  Jogos no ensino.	
	\end{abstract}


	\selectlanguage{english}
	\begin{abstract}
		This article aims to show the efficiency of the use of digital games in teaching. For him we developed a game called \ textit {Run Aedes} that aims to inform about \ textit {Aedes Aegypti}. We apply the game to a school in our city and we collect some important and curious data about the behavior of people (young people) in the use of the software. \\
			
		\textbf{Palavras-chave}: Educational games; Aedes Aegypti; Digital games; Games in education.	
	\end{abstract}
	
	
			
			
	\section{Introducao}%Na introdução, deve-se apresentar o tema do artigo e a problemática em que se insere. %Também se deve apresentar como a pesquisa foi realizada para discussão do item-problema.
	
	Da Antiguidade até o início do século XIX, predomina na prática-escolar uma aprendizagem de tipo passivo e receptivo. Aprender era quase exclusivamente memorizar. Neste tipo de aprendizagem, a compreensão desempenhava um papel muito reduzido(Eulina Castro de Souza, Ilana Castro de Souz, Vany Regina Teixeira).\\
	
	Os conhecimentos a serem adquiridos eram, até certo ponto, reduzidos. E para que os alunos pudessem repetí-los correta e adequadamente, o professor utilizava o procedimento de perguntas e respostas, tanto em sua forma oral como escrita. Este era o chamado método catequético, cuja origem remota, pelo menos cultura ocidental, aos antigos gregos. A palavra catecismo provém do termo grego katechein, que significa "fazer eco". Este método era usado por todas as disciplinas e consistia na apresentação, pelo professor, de perguntas acompanhadas de suas respostas já prontas.(Eulina Castro de Souza, Ilana Castro de Souz, Vany Regina Teixeira)\\
	
	Ao ensinar um assunto o professor deve:\\
	
	Apresentar o objeto ou idéia diretamente, fazendo demonstração, pois o aluno aprende através dos sentidos, principalmente vendo e tocando.\\
	
	Mostrar a utilidade específica do conhecimento transmitido e a sua aplicação na vida diária.
	Fazer referência à natureza e origem dos fenômenos estudados, isto é, às suas causas.\\
	
	Explicar primeiramente os princípios gerais e só depois os detalhes.\\
	
	Passar para o assunto ou tópico seguinte do conteúdo apenas quando o aluno tiver compreendido o anterior.(Eulina Castro de Souza, Ilana Castro de Souz, Vany Regina Teixeira)\\
	
	Com o decorrer do tempo esses métodos de ensino se tornam cada vez mais cansativos no olhar dos alunos. Com a tecnologia avançando e as pessoas cada vez mais imersas nesse mundo a forma de ensino não pode ficar parada no tempo,a tecnologia veio para nos auxiliar em tarefas consideradas dificeis isso inclui o ensino. Há um tempo atras as pessoas já vinham com idéias de melhorar o ensino com o uso de jogos, jogos educativos como: Blocos de montar, Quebra-cabeça, jogos de tabuleiro... todos tem como um objetivo a diversão, porém estimulam o raciocio logico desde de crianças até adultos, por isso a importancia de continuar investindo em jogos com uma forma de ensinar não só o raciocio logico,mas como qualquer tipo de área, desde materias escolares até topicos mais abstratos como comunidade, social, prevenções etc...
	
	Este artigo esta dividido em 4 partes incluindo a introdução, na primeira parte damos uma breve introdução sobre a problematica que estamos apresentando e dando algumas alternativas de solução, na segunda parte está o método utilizados, onde explicamos qual método utilizados para propor nossas ideias e o por que escolhemos essa técnica, na terceira parte vamos falar sobre a pesquisa, como a pesquisa foi elaborada, aplicada e analisada, na quarta parte terá nossa conclusão onde vamos repassar todos os conhecimentos e dados colhidos nessa pesquisa.
	
		
		
	\section{Método utilizado}% Qual método nois utilizamos para fazer essa pesquisa, no nosso caso foi uma %pesquisa aplicada
		Nosso Objetivo nesse artigo é demostrar a eficiência do uso de jogos para o ensino seja ele qual for. Desse modo  tivemos que ter algo que nos possibilitasse mostrar essa eficiência, por isso desenvolvemos o jogo \textbf{Run Aedes} para que pudéssemos aplicar o mesmo e colher as informações necessárias. Nesse âmbito utilizamos o método de pesquisa aplicada, juntamente com questionários, que poderia nos fornecer uma pesquisa eficiênte e rápida.
		
		A pesquisa aplicada tem como objetivo a utilização de toda informação disponível para a criação de novas tecnologias e métodos, transformando a sociedade atual em que vivemos. Esse tipo de pesquisa possui resultados mais palpáveis, muitas vezes percebidos pela população também.(Segundo o site Galoá journal).
		
		O uso de questionários na pesquisa nos forneceu um maior números de dados em pouco tempo, como aplicamos o jogo em uma escola (em uma sala para ser mais específico) não poderiamos interromper os alunos por muito tempo, então o uso do Questionário foi e extrema importância. No questionário teve perguntas tanto da jogabilidade como tambem sobre as informações passadas no jogo, se estavam claras, objetivas etc. Apesar de ter poucos participantes devido ao tempo, os dados coletados foram muito relevantes para a nossa pesquisa, os dados serão mostrados na parte 3 do artigo.
	
	
	
	\section{A pesquisa} % Nessa parte vamos falar desde a elaboração da pesquisa até a aplicação da mesma
	
	\subsection{Preparando pesquisa} %O que tinhamos que fazer para poder aplicar essa pesquisa
		A primeira coisa que tinhamos que fazer para poder aplicar essa pesquisa era uma versão jogavél do aplicativo, tivemos em média 2 meses para desenvolver o jogo do zero, desde a sua concepção até a sua publicação (publicamos na play store antes de fazer a pesquisa).Após o jogo está publicado começamos a preparar a pesquisa, primeiramente tivemos que fazer um termo de concentimento,pós esclarecimento e logo depois o questionario.\\
		
		O termo de concentimento é um documento onde informamos aos participantes por que estamos realizando essa pesquisa, quais os objetivos da pesquisa e quais os seus direitos ao esta participando da pesquisa, segue o anexo 1 \\
		
		pós esclarecimento é um documento onde o participante irá aceitar formalmente o termo de concentimento como tambem irá aceitar participar da pesquisa segue o anexo 2 \\
		
		O Questionario foi aplicado para ter um maior numero de dados em pouco tempo, pois como a pesquisa foi feita em uma escola não poderiamos interromper os alunos por muito tempo, por isso o questionario foi uma ferramente muito importante na nossa pesquisa. Nele havia perguntas desde a jogabilidade até as informações passadas no jogo, se as informações foram repassadas de forma eficiente e clara. segue o anexo 3\\
		
		
	\subsection{Aplicação do software}% O dia da aplicação da pesquisa
	\subsection{Dados coletados} % informações relevantes colhidas nessa pesquisa
	
	
	\section{Conclusão}
	\subsection{Lições aprendidas}
		As considerações finais tratam do fechamento do tema, ainda que reconhecendo os limites do própior atigo para apontar soluções, podendo-se pontuar a necessidade de novas investigações.
	
	
	
	\section{Referências} 
	
	Links: 	(Artigo sobre uso de jogos no ensino de programação)http://www.br-ie.org/pub/index.php/sbie/article/view/3000/2511
			(Pesquisa aplicada)https://galoa.com.br/blog/pesquisa-basica-e-pesquisa-aplicada-o-que-sao-e-suas-importancias
			(Evolução do ensino)http://www2.seduc.mt.gov.br/-/evolucao-historica-do-processo-ensino-aprendizag-3
	
	
	
\end{document}

