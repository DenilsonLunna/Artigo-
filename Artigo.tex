\documentclass[12pt]{article}
\usepackage[top=3cm,bottom=2cm,left=3cm,right=2cm]{geometry}
\usepackage[utf8]{inputenc}
\usepackage[brazilian]{babel}




\begin{document}
	
	\title{\textbf{Uso de jogos digitais para ensino}}
	\author{\textit{Denilson Sousa}}
	
	\maketitle % Data
	

	\abstract{
			Este artigo tem como objetivo mostrar a eficiência no uso de jogos digitais no ensino. Para ele desenvolvemos um jogo chamado \textit{Run Aedes} que tem como objetivo informar sobre o \textit{Aedes Aegypti}. Aplicamos o jogo em uma escola da nossa cidade e colhemos alguns dados importantes e curiosos sobre o comportamento das pessoas (jovens) no uso do software.\\
			
			\textbf{Palavras-chave}: Jogos Educativos; Aedes Aegypti; Jogos Digitais;  Jogos no ensino.
			
			
	\section{Introducao}
		Na introdução, deve-se apresentar o tema do artigo e a problemática em que se insere. Também se deve apresentar como a pesquisa foi realizada para discussão do tem-problema.
	
	\section{O projeto}
		No desenvolvimento e em seus subitens, discorre-se sobre  a questão envolvida no tema, recorrendo ás referências teóricas levantadas durante a pesquisa.
	\subsection{A Ideia}
	\subsection{O Desenvolvimento}
	\subsection{Problemas no desenvolvimento}
	
	
	\section{A pesquisa}
	
	\subsection{Preparando pesquisa}
	\subsection{Aplicação do software}
	\subsection{Dados coletados}
	
	
	\section{Conclusão}
		As considerações finais tratam do fechamento do tema, ainda que reconhecendo os limites do própior atigo para apontar soluções, podendo-se pontuar a necessidade de novas investigações.
	} 
	
	
		\section{Referências} 
	
	
	
\end{document}

